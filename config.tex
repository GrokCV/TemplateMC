% config.tex
% 指定 XeLaTeX 编译
%!TeX program = xelatex

%%%%%%%%%%%%%%%%%%%%%%%%%%%%%%%%%%%%%%%%%%%%%%%%%%%%%%%%%%
% 文档类与全局设置
%%%%%%%%%%%%%%%%%%%%%%%%%%%%%%%%%%%%%%%%%%%%%%%%%%%%%%%%%%
\documentclass{beamer}
%tikz包
\usepackage{amsmath, amssymb, xcolor, tikz}
\usetikzlibrary{calc}  % 计算坐标
\usetikzlibrary{backgrounds}  % 背景支持(如果想使用 on background layer)
\usepackage[most]{tcolorbox}
\usepackage{xparse}

% 采用 XeLaTeX 时加载 fontspec 与 xeCJK,实现中西文字体统一管理
\usepackage{fontspec}
\usepackage{xeCJK}
\usepackage{fontspec}
\usepackage{unicode-math}

% 设置英文字体
\setmainfont{Liberation Sans} 
\setsansfont{Liberation Sans}
\setmonofont{Liberation Mono}
% 设置数学公式字体
\setmathfont{TeX Gyre Termes Math} % 数学部分使用 New Roman 风格字体

% 设置中文字体(可根据需求调整)
\usepackage{xeCJK}
\setCJKmainfont{Noto Sans CJK SC} % 主要中文黑体
\setCJKsansfont{Noto Sans CJK SC}
\setCJKmonofont{Noto Sans CJK SC}
% 如果不需要区分宋体或黑体,相关宏定义可以省略

%%%%%%%%%%%%%%%%%%%%%%%%%%%%%%%%%%%%%%%%%%%%%%%%%%%%%%%%%%
% 图形、色彩与数学设置
%%%%%%%%%%%%%%%%%%%%%%%%%%%%%%%%%%%%%%%%%%%%%%%%%%%%%%%%%%
\usepackage{tikz}
\usepackage[table]{xcolor}    % 表格色块
\usepackage{booktabs}         % 三线表格式
\usepackage{pgfplots}
\pgfplotsset{compat=1.18}

\usepackage{tcolorbox}
\usepackage{amsmath}
\usepackage{xcolor}

\usepackage{unicode-math}
\setmathfont{TeX Gyre Termes Math}
\setbeamercolor{block title}{fg=black,bg=gray!10}
\setbeamerfont{block title}{size=\normalsize,family=\rmfamily}


%%%%%%%%%%%%%%%%%%%%%%%%%%%%%%%%%%%%%%%%%%%%%%%%%%%%%%%%%%
% Beamer 主题与风格设置
%%%%%%%%%%%%%%%%%%%%%%%%%%%%%%%%%%%%%%%%%%%%%%%%%%%%%%%%%%
% 清除默认页眉与 frametitle 模板(后续自定义)
\setbeamertemplate{headline}{}
\setbeamertemplate{frametitle}{}
\setbeamertemplate{itemize item}[circle]

% 内外主题(可根据需求替换)
\useoutertheme[subsection=true]{miniframes} 
\useinnertheme{default}
\usecolortheme{default}

%%%%%%%%%%%%%%%%%%%%%%%%%%%%%%%%%%%%%%%%%%%%%%%%%%%%%%%%%%
% 颜色定义(可自定义调色板)
%%%%%%%%%%%%%%%%%%%%%%%%%%%%%%%%%%%%%%%%%%%%%%%%%%%%%%%%%%
\definecolor{lightblue}{RGB}{70,130,180}
\definecolor{midblue}{RGB}{28,70,130}
\definecolor{darkblue}{RGB}{10,40,80}
\definecolor{myred}{RGB}{150,50,30}
\definecolor{mygreen}{RGB}{45,125,77}
\definecolor{mypurple}{RGB}{120,61,137}
\definecolor{myorange}{RGB}{190,88,32}
\definecolor{myitem}{RGB}{35,35,35}


%%%%%%%%%%%%%%%%%%%%%%%%%%%%%%%%%%%%%%%%%%%%%%%%%%%%%%%%%%
% 导航栏与页眉设置
%%%%%%%%%%%%%%%%%%%%%%%%%%%%%%%%%%%%%%%%%%%%%%%%%%%%%%%%%%
\makeatletter
% 设置顶部导航栏(用于章节、子章节导航)
\setbeamertemplate{headline}{%
  \begin{beamercolorbox}[wd=\paperwidth,ht=2ex,dp=1ex]{section in head/foot}
    \insertsectionnavigationhorizontal{\paperwidth}{}{}%
  \end{beamercolorbox}%
}
\setbeamercolor{section in head/foot}{bg=darkblue, fg=white}
\setbeamercolor{subsection in head/foot}{bg=midblue, fg=white}

% 自定义页眉(仅在非封面页显示 frametitle 和横线,且 frametitle 不为空时才显示)
\setbeamertemplate{headline}{%
  \ifnum\value{framenumber}>1
    \ifx\insertframetitle\@empty
      % 不显示任何内容
    \else
      \begin{tikzpicture}[remember picture,overlay]
        \node[anchor=west, font=\normalsize, text=black, xshift=0.2cm, yshift=-0.4cm]
              at (current page.north west){\insertframetitle};
        \draw[line width=1pt, color=midblue]
              ([yshift=-0.7cm]current page.north west) -- ([yshift=-0.7cm]current page.north east);
      \end{tikzpicture}
    \fi
  \fi
}
\makeatother
%%%%%%%%%%%%%%%%%%%%%%%%%%%%%%%%%%%%%%%%%%%%%%%%%%%%%%%%%%
% 图注样式设置
%%%%%%%%%%%%%%%%%%%%%%%%%%%%%%%%%%%%%%%%%%%%%%%%%%%%%%%%%%
\setbeamerfont{caption}{size=\footnotesize}
\setbeamercolor{caption}{fg=black}
% 此处使用默认字体,确保中文统一为 Noto Sans CJK SC
\setbeamerfont{caption name}{size=\footnotesize}
\setbeamercolor{caption name}{fg=black}
\setbeamertemplate{caption}[numbered]
\renewcommand{\figurename}{图}

%%%%%%%%%%%%%%%%%%%%%%%%%%%%%%%%%%%%%%%%%%%%%%%%%%%%%%%%%%
% 列表样式设置
%%%%%%%%%%%%%%%%%%%%%%%%%%%%%%%%%%%%%%%%%%%%%%%%%%%%%%%%%%
% 自定义无序列表图标,采用 \raisebox 微调图标位置
\newcommand{\useBigCircleItem}{%
  \setbeamertemplate{itemize item}{\raisebox{0.3ex}{\tikz{\draw[myitem, fill=myitem] (0,0) circle (2pt);}}}%
  \setbeamertemplate{itemize subitem}{\raisebox{0.3ex}{\tikz{\draw[myitem, fill=myitem] (0,0) circle (2pt);}}}%
  \setbeamertemplate{itemize subsubitem}{\raisebox{0.3ex}{\tikz{\draw[myitem, fill=myitem] (0,0) circle (2pt);}}}%
}
\newcommand{\useSmallCircleItem}{%
  \setbeamertemplate{itemize item}{\raisebox{0.35ex}{\tikz{\draw[myitem, fill=myitem, opacity=0.8] (0,0) circle (1pt);}}}%
  \setbeamertemplate{itemize subitem}{\raisebox{0.35ex}{\tikz{\draw[myitem, fill=myitem, opacity=0.8] (0,0) circle (1pt);}}}%
  \setbeamertemplate{itemize subsubitem}{\raisebox{0.35ex}{\tikz{\draw[myitem, fill=myitem, opacity=0.8] (0,0) circle (1pt);}}}%
}
% 默认使用大圆圈
\useBigCircleItem

% 有序列表样式
\setbeamercolor{enumerate item}{fg=darkblue}
\setbeamertemplate{enumerate item}{%
  \usebeamerfont*{enumerate item}%
  \color{enumerate item.fg}\insertenumlabel.%
}

%%%%%%%%%%%%%%%%%%%%%%%%%%%%%%%%%%%%%%%%%%%%%%%%%%%%%%%%%%
% 页脚设置
%%%%%%%%%%%%%%%%%%%%%%%%%%%%%%%%%%%%%%%%%%%%%%%%%%%%%%%%%%
\setbeamertemplate{footline}{%
  \begin{tikzpicture}[remember picture,overlay]
    \node[fill=lightblue, minimum width=0.3\paperwidth, minimum height={0.4cm+1pt}, anchor=south west]
         at ([xshift=-1pt,yshift=-1pt]current page.south west) (leftblock) {};
    \node[fill=midblue, minimum width=0.6\paperwidth, minimum height={0.4cm+1pt}, anchor=south west]
         at ([xshift=0.3\paperwidth-1pt,yshift=-1pt]current page.south west) (midblock) {};
    \node[fill=darkblue, minimum width={0.1\paperwidth+1pt}, minimum height={0.4cm+1pt}, anchor=south west]
         at ([xshift=0.9\paperwidth-1pt,yshift=-1pt]current page.south west) (rightblock) {};
    \node[anchor=center, yshift=0.5pt, font=\tiny, text=white] at (leftblock.center)
         {程明明, http://mmcheng.net};
    \node[anchor=center, yshift=0.5pt, font=\tiny, text=white] at (midblock.center)
         {大规模图像的多粒度目标检测};
    \node[anchor=center, yshift=0.5pt, font=\tiny, text=white] at (rightblock.center)
         {\insertframenumber};
  \end{tikzpicture}
}

%%%%%%%%%%%%%%%%%%%%%%%%%%%%%%%%%%%%%%%%%%%%%%%%%%%%%%%%%%
% 注释宏定义
%%%%%%%%%%%%%%%%%%%%%%%%%%%%%%%%%%%%%%%%%%%%%%%%%%%%%%%%%%
\newcounter{cornernotecounter}
\newcommand{\cornernote}[1]{%
  \begin{tikzpicture}[remember picture, overlay]
    \node[anchor=south west, xshift=0.6cm, yshift=0.45cm] at (current page.south west) {%
      {\scriptsize\color{gray}\textsuperscript{\thecornernotecounter} #1}%
    };
  \end{tikzpicture}%
}

\newcommand{\cornerfootnote}[1]{%
  \stepcounter{cornernotecounter}%
  \textsuperscript{\thecornernotecounter}%
  \cornernote{#1}%
}

%%%%%%%%%%%%%%%%%%%%%%%%%%%%%%%%%%%%%%%%%%%%%%%%%%%%%%%%%%
% 表格样式宏定义
%%%%%%%%%%%%%%%%%%%%%%%%%%%%%%%%%%%%%%%%%%%%%%%%%%%%%%%%%%
\newcommand{\PlainTable}[1]{%
  {\footnotesize
   \setlength{\tabcolsep}{6pt}%
   #1%
  }%
}
\newcommand{\ThreeLineTable}[1]{%
  {\footnotesize
   \setlength{\tabcolsep}{6pt}%
   #1%
  }%
}
\newcommand{\AltColorTable}[1]{%
  {\footnotesize
   \setlength{\tabcolsep}{6pt}%
   \rowcolors{2}{gray!20}{white}%
   #1%
   \rowcolors{2}{}{}%
  }%
}

%%%%%%%%%%%%%%%%%%%%%%%%%%%%%%%%%%%%%%%%%%%%%%%%%%%%%%%%%%
%强调块(适用于公式等)
%%%%%%%%%%%%%%%%%%%%%%%%%%%%%%%%%%%%%%%%%%%%%%%%%%%%%%%%%%
% 定义参数 key 和默认值
\pgfkeys{
  /BehindEqBox/.is family, /BehindEqBox,
  color/.initial = midblue,
  border/.initial = 1pt,
  style/.initial = solid,
  padding/.initial = 2.8mm
}

% 定义命令:支持键值对参数 + 内容
\NewDocumentCommand{\BehindEqBox}{O{} m}{%
  % 加载参数
  \pgfkeys{/BehindEqBox, #1}

  \vspace{1em}
  \begin{center}
    \begin{tikzpicture}[remember picture, overlay]
      \node[inner sep=\pgfkeysvalueof{/BehindEqBox/padding}] (E) {#2};

      \coordinate (A) at ([xshift=-25pt]E.south west);
      \coordinate (B) at ([xshift=25pt]E.north east);

      \begin{pgfonlayer}{background}
        \filldraw[fill=\pgfkeysvalueof{/BehindEqBox/color}, opacity=0.1, rounded corners=0.35cm]
          (A) rectangle (B);

        \draw[
          line width=\pgfkeysvalueof{/BehindEqBox/border},
          draw=\pgfkeysvalueof{/BehindEqBox/color},
          opacity=0.2,
          rounded corners=0.35cm,
          \pgfkeysvalueof{/BehindEqBox/style}
        ] 
        (A) rectangle (B);
      \end{pgfonlayer}
    \end{tikzpicture}
  \end{center}
  \vspace{1em}
}


%%%%%%%%%%%%%%%%%%%%%%%%%%%%%%%%%%%%%%%%%%%%%%%%%%%%%%%%%%
%字体高亮块
%%%%%%%%%%%%%%%%%%%%%%%%%%%%%%%%%%%%%%%%%%%%%%%%%%%%%%%%%%
% 替换原 block 环境
\newtcolorbox{beamerblockbox}[3][]{%
  enhanced,
  breakable,
  colback=#2!10,             % 内容背景(透明度 80%)
  colframe=#2,               % 框颜色 = 标题栏颜色
  coltitle=white,            % 标题文字颜色
  fonttitle=\large,
  title=#3,
  boxed title style={
    colback=#2,
    top=2pt, bottom=2pt,
    left=2pt, right=2pt,
    sharp corners=northwest,
    sharp corners=northeast,
  },
  boxrule=0pt,
  arc=1.5mm,
  shadow={0pt}{-1pt}{1.5mm}{black!30},
  left=2pt, right=2pt,
  top=2pt, bottom=2pt,
  before skip=6pt,  % block与上文之间的空隙
  #1,
}

% 替换默认 beamer block 环境
\let\oldblock\block
\renewenvironment{block}[1]{\begin{beamerblockbox}{midblue}{#1}}{\end{beamerblockbox}}

\let\oldalertblock\alertblock
\renewenvironment{alertblock}[1]{\begin{beamerblockbox}{myred}{#1}}{\end{beamerblockbox}}

\let\oldexampleblock\exampleblock
\renewenvironment{exampleblock}[1]{\begin{beamerblockbox}{myorange}{#1}}{\end{beamerblockbox}}
